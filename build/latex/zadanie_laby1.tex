%% Generated by Sphinx.
\def\sphinxdocclass{report}
\documentclass[letterpaper,10pt,polish]{sphinxmanual}
\ifdefined\pdfpxdimen
   \let\sphinxpxdimen\pdfpxdimen\else\newdimen\sphinxpxdimen
\fi \sphinxpxdimen=.75bp\relax
\ifdefined\pdfimageresolution
    \pdfimageresolution= \numexpr \dimexpr1in\relax/\sphinxpxdimen\relax
\fi
%% let collapsible pdf bookmarks panel have high depth per default
\PassOptionsToPackage{bookmarksdepth=5}{hyperref}

\PassOptionsToPackage{booktabs}{sphinx}
\PassOptionsToPackage{colorrows}{sphinx}

\PassOptionsToPackage{warn}{textcomp}
\usepackage[utf8]{inputenc}
\ifdefined\DeclareUnicodeCharacter
% support both utf8 and utf8x syntaxes
  \ifdefined\DeclareUnicodeCharacterAsOptional
    \def\sphinxDUC#1{\DeclareUnicodeCharacter{"#1}}
  \else
    \let\sphinxDUC\DeclareUnicodeCharacter
  \fi
  \sphinxDUC{00A0}{\nobreakspace}
  \sphinxDUC{2500}{\sphinxunichar{2500}}
  \sphinxDUC{2502}{\sphinxunichar{2502}}
  \sphinxDUC{2514}{\sphinxunichar{2514}}
  \sphinxDUC{251C}{\sphinxunichar{251C}}
  \sphinxDUC{2572}{\textbackslash}
\fi
\usepackage{cmap}
\usepackage[T1]{fontenc}
\usepackage{amsmath,amssymb,amstext}
\usepackage{babel}



\usepackage{tgtermes}
\usepackage{tgheros}
\renewcommand{\ttdefault}{txtt}



\usepackage[Sonny]{fncychap}
\ChNameVar{\Large\normalfont\sffamily}
\ChTitleVar{\Large\normalfont\sffamily}
\usepackage{sphinx}

\fvset{fontsize=auto}
\usepackage{geometry}


% Include hyperref last.
\usepackage{hyperref}
% Fix anchor placement for figures with captions.
\usepackage{hypcap}% it must be loaded after hyperref.
% Set up styles of URL: it should be placed after hyperref.
\urlstyle{same}

\addto\captionspolish{\renewcommand{\contentsname}{Contents:}}

\usepackage{sphinxmessages}
\setcounter{tocdepth}{1}



\title{zadanie\_laby1}
\date{04 lis 2025}
\release{1}
\author{290306}
\newcommand{\sphinxlogo}{\vbox{}}
\renewcommand{\releasename}{Wydanie}
\makeindex
\begin{document}

\ifdefined\shorthandoff
  \ifnum\catcode`\=\string=\active\shorthandoff{=}\fi
  \ifnum\catcode`\"=\active\shorthandoff{"}\fi
\fi

\pagestyle{empty}
\sphinxmaketitle
\pagestyle{plain}
\sphinxtableofcontents
\pagestyle{normal}
\phantomsection\label{\detokenize{index::doc}}


\sphinxAtStartPar
Add your content using \sphinxcode{\sphinxupquote{reStructuredText}} syntax. See the
\sphinxhref{https://www.sphinx-doc.org/en/master/usage/restructuredtext/index.html}{reStructuredText}
documentation for details.

\sphinxstepscope


\chapter{Wprowadzenie}
\label{\detokenize{rozdzial1/index:wprowadzenie}}\label{\detokenize{rozdzial1/index::doc}}
\sphinxAtStartPar
To jest wiele razy powturzony opis bigosu sens egzystencji i cel istnienia


\section{BIGOS}
\label{\detokenize{rozdzial1/index:bigos}}
\sphinxAtStartPar
W kociołkach bigos grzano; w słowach wydać trudno
Bigosu smak przedziwny, kolor i woń cudną;
Słów tylko brzęk usłyszy i rymów porządek,
Ale treści ich miejski nie pojmie żołądek.
Aby cenić litewskie pieśni i potrawy,
Trzeba mieć zdrowie, na wsi żyć, wracać z obławy.

\sphinxAtStartPar
Przecież i bez tych przypraw potrawą nie lada
Jest bigos, bo się z jarzyn dobrych sztucznie składa.
Bierze się doń siekana, kwaszona kapusta,
Która, wedle przysłowia, sama idzie w usta;
Zamknięta w kotle, łonem wilgotnym okrywa
Wyszukanego cząstki najlepsze mięsiwa;
I praży się, aż ogień wszystkie z niej wyciśnie
Soki żywne, aż z brzegów naczynia war pryśnie
I powietrze dokoła zionie aromatem.

\sphinxAtStartPar
Bigos już gotów. Strzelcy z trzykrotnym wiwatem,
Zbrojni łyżkami, biegą i bodą naczynie,
Miedź grzmi, dym bucha, bigos jak kamfora ginie,
Zniknął, uleciał; tylko w czeluściach saganów
Wre para, jak w kraterze zagasłych wulkanów.

\sphinxAtStartPar
W kociołkach bigos grzano; w słowach wydać trudno
Bigosu smak przedziwny, kolor i woń cudną;
Słów tylko brzęk usłyszy i rymów porządek,
Ale treści ich miejski nie pojmie żołądek.
Aby cenić litewskie pieśni i potrawy,
Trzeba mieć zdrowie, na wsi żyć, wracać z obławy.

\sphinxAtStartPar
Przecież i bez tych przypraw potrawą nie lada
Jest bigos, bo się z jarzyn dobrych sztucznie składa.
Bierze się doń siekana, kwaszona kapusta,
Która, wedle przysłowia, sama idzie w usta;
Zamknięta w kotle, łonem wilgotnym okrywa
Wyszukanego cząstki najlepsze mięsiwa;
I praży się, aż ogień wszystkie z niej wyciśnie
Soki żywne, aż z brzegów naczynia war pryśnie
I powietrze dokoła zionie aromatem.

\sphinxAtStartPar
Bigos już gotów. Strzelcy z trzykrotnym wiwatem,
Zbrojni łyżkami, biegą i bodą naczynie,
Miedź grzmi, dym bucha, bigos jak kamfora ginie,
Zniknął, uleciał; tylko w czeluściach saganów
Wre para, jak w kraterze zagasłych wulkanów.

\sphinxAtStartPar
W kociołkach bigos grzano; w słowach wydać trudno
Bigosu smak przedziwny, kolor i woń cudną;
Słów tylko brzęk usłyszy i rymów porządek,
Ale treści ich miejski nie pojmie żołądek.
Aby cenić litewskie pieśni i potrawy,
Trzeba mieć zdrowie, na wsi żyć, wracać z obławy.

\sphinxAtStartPar
Przecież i bez tych przypraw potrawą nie lada
Jest bigos, bo się z jarzyn dobrych sztucznie składa.
Bierze się doń siekana, kwaszona kapusta,
Która, wedle przysłowia, sama idzie w usta;
Zamknięta w kotle, łonem wilgotnym okrywa
Wyszukanego cząstki najlepsze mięsiwa;
I praży się, aż ogień wszystkie z niej wyciśnie
Soki żywne, aż z brzegów naczynia war pryśnie
I powietrze dokoła zionie aromatem.

\sphinxAtStartPar
Bigos już gotów. Strzelcy z trzykrotnym wiwatem,
Zbrojni łyżkami, biegą i bodą naczynie,
Miedź grzmi, dym bucha, bigos jak kamfora ginie,
Zniknął, uleciał; tylko w czeluściach saganów
Wre para, jak w kraterze zagasłych wulkanów.

\sphinxAtStartPar
W kociołkach bigos grzano; w słowach wydać trudno
Bigosu smak przedziwny, kolor i woń cudną;
Słów tylko brzęk usłyszy i rymów porządek,
Ale treści ich miejski nie pojmie żołądek.
Aby cenić litewskie pieśni i potrawy,
Trzeba mieć zdrowie, na wsi żyć, wracać z obławy.

\sphinxAtStartPar
Przecież i bez tych przypraw potrawą nie lada
Jest bigos, bo się z jarzyn dobrych sztucznie składa.
Bierze się doń siekana, kwaszona kapusta,
Która, wedle przysłowia, sama idzie w usta;
Zamknięta w kotle, łonem wilgotnym okrywa
Wyszukanego cząstki najlepsze mięsiwa;
I praży się, aż ogień wszystkie z niej wyciśnie
Soki żywne, aż z brzegów naczynia war pryśnie
I powietrze dokoła zionie aromatem.

\sphinxAtStartPar
Bigos już gotów. Strzelcy z trzykrotnym wiwatem,
Zbrojni łyżkami, biegą i bodą naczynie,
Miedź grzmi, dym bucha, bigos jak kamfora ginie,
Zniknął, uleciał; tylko w czeluściach saganów
Wre para, jak w kraterze zagasłych wulkanów.

\sphinxAtStartPar
W kociołkach bigos grzano; w słowach wydać trudno
Bigosu smak przedziwny, kolor i woń cudną;
Słów tylko brzęk usłyszy i rymów porządek,
Ale treści ich miejski nie pojmie żołądek.
Aby cenić litewskie pieśni i potrawy,
Trzeba mieć zdrowie, na wsi żyć, wracać z obławy.

\sphinxAtStartPar
Przecież i bez tych przypraw potrawą nie lada
Jest bigos, bo się z jarzyn dobrych sztucznie składa.
Bierze się doń siekana, kwaszona kapusta,
Która, wedle przysłowia, sama idzie w usta;
Zamknięta w kotle, łonem wilgotnym okrywa
Wyszukanego cząstki najlepsze mięsiwa;
I praży się, aż ogień wszystkie z niej wyciśnie
Soki żywne, aż z brzegów naczynia war pryśnie
I powietrze dokoła zionie aromatem.

\sphinxAtStartPar
Bigos już gotów. Strzelcy z trzykrotnym wiwatem,
Zbrojni łyżkami, biegą i bodą naczynie,
Miedź grzmi, dym bucha, bigos jak kamfora ginie,
Zniknął, uleciał; tylko w czeluściach saganów
Wre para, jak w kraterze zagasłych wulkanów.

\sphinxAtStartPar
W kociołkach bigos grzano; w słowach wydać trudno
Bigosu smak przedziwny, kolor i woń cudną;
Słów tylko brzęk usłyszy i rymów porządek,
Ale treści ich miejski nie pojmie żołądek.
Aby cenić litewskie pieśni i potrawy,
Trzeba mieć zdrowie, na wsi żyć, wracać z obławy.

\sphinxAtStartPar
Przecież i bez tych przypraw potrawą nie lada
Jest bigos, bo się z jarzyn dobrych sztucznie składa.
Bierze się doń siekana, kwaszona kapusta,
Która, wedle przysłowia, sama idzie w usta;
Zamknięta w kotle, łonem wilgotnym okrywa
Wyszukanego cząstki najlepsze mięsiwa;
I praży się, aż ogień wszystkie z niej wyciśnie
Soki żywne, aż z brzegów naczynia war pryśnie
I powietrze dokoła zionie aromatem.

\sphinxAtStartPar
Bigos już gotów. Strzelcy z trzykrotnym wiwatem,
Zbrojni łyżkami, biegą i bodą naczynie,
Miedź grzmi, dym bucha, bigos jak kamfora ginie,
Zniknął, uleciał; tylko w czeluściach saganów
Wre para, jak w kraterze zagasłych wulkanów.

\sphinxAtStartPar
W kociołkach bigos grzano; w słowach wydać trudno
Bigosu smak przedziwny, kolor i woń cudną;
Słów tylko brzęk usłyszy i rymów porządek,
Ale treści ich miejski nie pojmie żołądek.
Aby cenić litewskie pieśni i potrawy,
Trzeba mieć zdrowie, na wsi żyć, wracać z obławy.

\sphinxAtStartPar
Przecież i bez tych przypraw potrawą nie lada
Jest bigos, bo się z jarzyn dobrych sztucznie składa.
Bierze się doń siekana, kwaszona kapusta,
Która, wedle przysłowia, sama idzie w usta;
Zamknięta w kotle, łonem wilgotnym okrywa
Wyszukanego cząstki najlepsze mięsiwa;
I praży się, aż ogień wszystkie z niej wyciśnie
Soki żywne, aż z brzegów naczynia war pryśnie
I powietrze dokoła zionie aromatem.

\sphinxAtStartPar
Bigos już gotów. Strzelcy z trzykrotnym wiwatem,
Zbrojni łyżkami, biegą i bodą naczynie,
Miedź grzmi, dym bucha, bigos jak kamfora ginie,
Zniknął, uleciał; tylko w czeluściach saganów
Wre para, jak w kraterze zagasłych wulkanów.

\sphinxAtStartPar
W kociołkach bigos grzano; w słowach wydać trudno
Bigosu smak przedziwny, kolor i woń cudną;
Słów tylko brzęk usłyszy i rymów porządek,
Ale treści ich miejski nie pojmie żołądek.
Aby cenić litewskie pieśni i potrawy,
Trzeba mieć zdrowie, na wsi żyć, wracać z obławy.

\sphinxAtStartPar
Przecież i bez tych przypraw potrawą nie lada
Jest bigos, bo się z jarzyn dobrych sztucznie składa.
Bierze się doń siekana, kwaszona kapusta,
Która, wedle przysłowia, sama idzie w usta;
Zamknięta w kotle, łonem wilgotnym okrywa
Wyszukanego cząstki najlepsze mięsiwa;
I praży się, aż ogień wszystkie z niej wyciśnie
Soki żywne, aż z brzegów naczynia war pryśnie
I powietrze dokoła zionie aromatem.

\sphinxAtStartPar
Bigos już gotów. Strzelcy z trzykrotnym wiwatem,
Zbrojni łyżkami, biegą i bodą naczynie,
Miedź grzmi, dym bucha, bigos jak kamfora ginie,
Zniknął, uleciał; tylko w czeluściach saganów
Wre para, jak w kraterze zagasłych wulkanów.

\sphinxAtStartPar
W kociołkach bigos grzano; w słowach wydać trudno
Bigosu smak przedziwny, kolor i woń cudną;
Słów tylko brzęk usłyszy i rymów porządek,
Ale treści ich miejski nie pojmie żołądek.
Aby cenić litewskie pieśni i potrawy,
Trzeba mieć zdrowie, na wsi żyć, wracać z obławy.

\sphinxAtStartPar
Przecież i bez tych przypraw potrawą nie lada
Jest bigos, bo się z jarzyn dobrych sztucznie składa.
Bierze się doń siekana, kwaszona kapusta,
Która, wedle przysłowia, sama idzie w usta;
Zamknięta w kotle, łonem wilgotnym okrywa
Wyszukanego cząstki najlepsze mięsiwa;
I praży się, aż ogień wszystkie z niej wyciśnie
Soki żywne, aż z brzegów naczynia war pryśnie
I powietrze dokoła zionie aromatem.

\sphinxAtStartPar
Bigos już gotów. Strzelcy z trzykrotnym wiwatem,
Zbrojni łyżkami, biegą i bodą naczynie,
Miedź grzmi, dym bucha, bigos jak kamfora ginie,
Zniknął, uleciał; tylko w czeluściach saganów
Wre para, jak w kraterze zagasłych wulkanów.

\sphinxstepscope


\chapter{Tabela i lista}
\label{\detokenize{rozdzial2/index:tabela-i-lista}}\label{\detokenize{rozdzial2/index::doc}}\begin{itemize}
\item {} \begin{description}
\sphinxlineitem{lista}\begin{itemize}
\item {} 
\sphinxAtStartPar
lista

\item {} 
\sphinxAtStartPar
lista

\end{itemize}

\end{description}

\end{itemize}
\begin{enumerate}
\sphinxsetlistlabels{\arabic}{enumi}{enumii}{}{.}%
\item {} 
\sphinxAtStartPar
lista
\begin{enumerate}
\sphinxsetlistlabels{\arabic}{enumii}{enumiii}{}{.}%
\item {} 
\sphinxAtStartPar
lista

\item {} 
\sphinxAtStartPar
lsita

\end{enumerate}

\end{enumerate}


\begin{savenotes}\sphinxattablestart
\sphinxthistablewithglobalstyle
\centering
\begin{tabulary}{\linewidth}[t]{TTT}
\sphinxtoprule
\sphinxstyletheadfamily 
\sphinxAtStartPar
co?
&\sphinxstyletheadfamily 
\sphinxAtStartPar
Blee
&\sphinxstyletheadfamily 
\sphinxAtStartPar
mniam
\\
\sphinxmidrule
\sphinxtableatstartofbodyhook
\sphinxAtStartPar
bigos bigos m
&
\sphinxAtStartPar
nie
&
\sphinxAtStartPar
tak
\\
\sphinxhline
\sphinxAtStartPar
bigos bigos k
&
\sphinxAtStartPar
nie
&
\sphinxAtStartPar
tak
\\
\sphinxhline
\sphinxAtStartPar
bigos bigos r
&
\sphinxAtStartPar
nie
&
\sphinxAtStartPar
tak
\\
\sphinxhline
\sphinxAtStartPar
bigos bigos l
&
\sphinxAtStartPar
nie
&
\sphinxAtStartPar
tak
\\
\sphinxbottomrule
\end{tabulary}
\sphinxtableafterendhook\par
\sphinxattableend\end{savenotes}

\sphinxstepscope


\chapter{Zdjecie}
\label{\detokenize{rozdzial3/index:zdjecie}}\label{\detokenize{rozdzial3/index::doc}}
\noindent\sphinxincludegraphics{{kociak}.jpg}



\renewcommand{\indexname}{Indeks}
\printindex
\end{document}